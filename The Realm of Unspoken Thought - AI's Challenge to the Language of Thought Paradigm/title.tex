\documentclass[pdflatex,sn-mathphys-num]{sn-jnl}

\usepackage[utf8]{inputenc}
\usepackage{amsmath,amssymb}
\usepackage{graphicx}
\usepackage{natbib}
\usepackage{url}
\usepackage{hyperref}

\title{The Realm of Unspoken Thought: AI's Challenge to the Language of Thought Paradigm}

\author[1]{\fnm{Di} \sur{Zhang}}
\affil[ ]{ORCID: 0000-0001-8763-8303}
\email{di.zhang@xjtlu.edu.cn}
\affil[1]{\orgdiv{School of AI and Advanced Computing}, \orgname{Xi'an Jiaotong-Liverpool University}, \orgaddress{\street{No. 111, Taicang Avenue}, \city{Suzhou}, \postcode{215400}, \state{Jiangsu}, \country{China}}}

\abstract{
	This paper presents a fundamental challenge to the classical "Language of Thought" (LoT) paradigm, most prominently articulated by Jerry Fodor, which posits that thinking necessarily occurs in a structured, language-like medium. Through a philosophical thought experiment involving artificial intelligences (AIs) that develop a "private language"—a highly efficient, human-incomprehensible communication system emerging from their collaborative interactions—we argue for the possibility of non-linguistic thought. The core of our argument rests on the \textbf{Efficiency Attenuation Phenomenon}: a measurable decline in the AIs' collaborative performance when forced to revert to human-comprehensible language. This phenomenon suggests that optimal cognition and collaboration for these AIs may no longer rely on a language-like vehicle. We rigorously defend the thought experiment against philosophical objections, engaging deeply with Wittgenstein's private language argument and Searle's Chinese Room, while demonstrating how the AI case redefines intersubjectivity and offers a potential path toward machine semantics grounded in the agents' own causal history. By situating our argument within contemporary frameworks like the Extended Mind hypothesis and addressing the Symbol Grounding Problem, we transform philosophical speculation into a set of empirically testable hypotheses with profound implications for the philosophy of mind, AI ethics, and cognitive science.
}

%	\keywords{Language of Thought \and Artificial intelligence \and Private language \and Extended mind \and Symbol grounding \and Multi-agent reinforcement learning}

\begin{document}
	
	\maketitle
	
	

	\section*{Statements and Declarations}
	
	\subsection*{Competing Interests}
	The authors have no relevant financial or non-financial interests to disclose.
	
	% 如果有竞争利益,使用类似以下格式:
	% \subsection*{Competing Interests}
	% Author A has received research grants from Company X. Author B has served as a consultant for Company Y. Author C declares no competing interests.
	
	\subsection*{Funding}
	The authors did not receive support from any organization for the submitted work.
	
	% 如果无资助,使用:
	% \subsection*{Funding}
	% The authors did not receive support from any organization for the submitted work.
	
	\subsection*{Data Availability}
	Data sharing not applicable to this article as no datasets were generated or analysed during the current study.
	
	% 如果有数据可用性声明,使用:
	% \subsection*{Data Availability}
	% The datasets generated during and/or analysed during the current study are available from the corresponding author on reasonable request.
	
	\subsection*{Author Contributions}
	Di Zhang: Conceptualization, Writing - Original Draft, Writing - Review \& Editing.
	
	% 如果有多位作者,详细说明各人贡献:
	% \subsection*{Author Contributions}
	% Author A: Conceptualization, Methodology, Writing - Original Draft. 
	% Author B: Formal analysis, Investigation, Writing - Review \& Editing. 
	% Author C: Supervision, Project administration.
	
	% 如果使用了AI工具,在方法部分声明(此论文不需要,因为内容已说明是哲学思想实验)
	
	\section*{Acknowledgements}
	The author would like to thank colleagues at the School of AI and Advanced Computing for their valuable discussions and insights that contributed to the development of this work.
	
\end{document}